\documentclass[a4paper,12pt]{article} % добавить leqno в [] для нумерации слева

%%% Работа с русским языком
\usepackage{cmap}					% поиск в PDF
\usepackage{mathtext} 				% русские буквы в фомулах
\usepackage[T2A]{fontenc}		  	% кодировка
\usepackage[utf8]{inputenc}			% кодировка исходного текста
\usepackage[english,russian]{babel}	% локализация и переносы
\usepackage[left=1.4cm,right=2cm,top=2cm,bottom=2cm,footskip=1cm,includefoot
] {geometry}
%%% Дополнительная работа с математикой
\usepackage{amsmath,amsfonts,amssymb,amsthm,mathtools} % AMS
\usepackage{icomma} % "Умная" запятая: $0,2$ --- число, $0, 2$ --- перечисление

%% Номера формул
\mathtoolsset{showonlyrefs=true} % Показывать номера только у тех формул, на которые есть \eqref{} в тексте.

%% Перенос знаков в формулах (по Львовскому)
\newcommand*{\hm}[1]{#1\nobreak\discretionary{}
	{\hbox{$\mathsurround=0pt #1$}}{}}

%%% Работа с картинками
\usepackage{graphicx}  % Для вставки рисунков
\graphicspath{{images/}{images2/}}  % папки с картинками
\setlength\fboxsep{3pt} % Отступ рамки \fbox{} от рисунка
\setlength\fboxrule{1pt} % Толщина линий рамки \fbox{}
\usepackage{wrapfig} % Обтекание рисунков и таблиц текстом

%%% Работа с таблицами
\usepackage{array,tabularx,tabulary,booktabs} % Дополнительная работа с таблицами

\usepackage{longtable}  % Длинные таблицы
\usepackage{multirow} % Слияние строк в таблице
\newcommand{\RomanNumeralCaps}[1]
{\MakeUppercase{\romannumeral #1}}
\newcommand{\tb}[1]{\textbf{#1}}
\newcommand{\abs}[1]{\left|#1\right|}
\newcommand{\ol}[1]{\ensuremath{\overline{#1}}}
\usepackage{bigints}
\usepackage{pgfplots, pgfplotstable}
\pgfplotsset{compat=1.9}
\usepackage{circuitikz}
\usepackage{ulem}
\usepackage{cancel}
\usepackage{tikz}
\usetikzlibrary{calc}
\usepackage{tkz-euclide}
\usepackage{multicol}
\usepackage{enumitem}
%\DeclareMathOperator{\arctg}{arctg}
\begin{document} % конец преамбулы, начало документа

\begin{figure}[h!]
	\hfill
	\includegraphics[height=.09\textheight]{sirius}
\end{figure}
\vspace{5ex}
\begin{center}
	\Large\textbf{Неопределенный интеграл.\\
		Общие приемы и методы интегрирования} 
\end{center}

\paragraph*{Определение.}Пусть 
$ -\infty\leqslant a<b\leqslant+\infty $ и на $ (a, b) $
заданы функции $ f(x) $ и $ F(x) $. Функция $ F(x) $
называется \textit{первообразной} функции $ f(x) $ на 
$ (a, b) $, если
\[
	\forall x\in(a, b)\hookrightarrow F'(x)=f(x).
\] 
\paragraph{Теорема 1.}(О структуре множества первообразных.)
Пусть функция $ F(x) $ является первообразной функции 
$ f(x) $ на $ (a, b) $. Тогда функция $ F(x)_1 $ является
первообразной функции $ f(x) $ на $ (a, b) $ в том и только
в том случае, если 
$ \exists C\in\mathbb{R}: \forall x\in(a, b)
\hookrightarrow F_1(x)=F(x)+C.$
\paragraph{Определение.}\textit{Неопределенным интегралом}
$ \displaystyle\int f(x)dx $ называется множество всех первообразных 
функции $ f(x) $.
\paragraph{Теорема 2.}Пусть функция $ F(x) $ является 
первообразной функции $ f(x) $. Тогда неопределенный 
интеграл функции $ f(x) $~--- это множество функций вида
$ F(x)+C $, где $ C\in\mathbb{R} $~--- произвольная константа:
$ \int f(x)dx=\{F(x)+C:C\in\mathbb{R}\} $, что для краткости
записывают в виде
\begin{equation}
	\int f(x)dx=F(x)+C.
\end{equation}
\paragraph{Замечание.} Операция взятия дифференциала $ d $
и операция взятия неопределенного интеграла 
$ \displaystyle\int $ являются взаимно обратными.
\paragraph{Теорема 3.} (Свойства линейности неопределенного
интеграла).Если функции $ f_1(x) $ и $ f_2(x) $ имеют
первообразные на $ (a, b) $, $ \alpha_1\in\mathbb{R} $, 
$ \alpha_2\in\mathbb{R} $, $ \alpha_1^2+\alpha_2^2\ne0 $, 
то на $ (a, b) $
\begin{equation}
	\int(\alpha_1f_1(x)+\alpha_2f_2(x))dx=
	\alpha_1\int f_1(x)dx+\alpha_2\int f_2(x)dx.
\end{equation}
\paragraph{Теорема 4.}(Метод интегрирования по частям.)
Пусть на $ (a, b) $ заданы дифференцируемые функции 
$ u(x) $ и $ v(x) $. Тогда на $ (a, b) $
\begin{equation}
	\int u(x)dv(x)=u(x)v(x)-\int v(x)du(x).
\end{equation}
\newpage
\paragraph{Теорема 5.}(Формулы для основных неопределенных
интегралов.)
\begin{enumerate}
\begin{multicols}{2}
		\item $ \displaystyle\int x^{\alpha}dx=
		\dfrac{x^{\alpha+1}}{\alpha+1}+C, ~~~ \alpha\ne1, x>0;$
		\item $ \displaystyle\int\dfrac{dx}{x+a}=
		\ln\abs{x+a}+C, ~~~ x\ne -a; $
		\item $ \displaystyle\int a^xdx=
		\dfrac{a^x}{\ln a}+C, ~~~ a>0, a\ne1; $
		\item $ \displaystyle\int\sin xdx=-\cos x+C; $
		\item $ \displaystyle\int\cos xdx=\sin x+C; $
		\item $ \displaystyle\int\dfrac{dx}{\cos^2x}=
		\tg x+C, x\ne\frac{\pi}{2}+\pi k; $
		\item $ \displaystyle\int\dfrac{dx}{x^2+a^2}=
		\frac{1}{a}\arctg\frac{x}{a}+C, ~~~ a>0; $
		\item $  \displaystyle\int\dfrac{dx}{\sqrt{a^2-x^2}}=
		\arcsin \frac{x}{a}+C, ~~~ \abs{x}<a;$
		\item $  \displaystyle\int\dfrac{dx}{x^2-a^2}=
		\frac{1}{2a}\ln\abs{\dfrac{x-a}{x+a}}+C,~~~x\ne\pm a;$
		\item 
		$  \displaystyle\int\dfrac{dx}{\sqrt{x^2+a}}=
		\ln\abs{x+\sqrt{x^2+a}}+C,~~x^2>-a. $
\end{multicols}
\end{enumerate}

\section*{Задачи для работы в классе}

Вычислить производную функции $ y=f(x) $. Указать область
существования производной.
\begin{enumerate}[label=\textbf{\arabic*.}]
	\item Найти какую-либо первообразную $ F(x) $ функции
	$ f(x)=1/\sqrt{x} $, $ x\in(0,+\infty) $, и ее
	неопределенный интеграл.
	\item Для функции $ f(x)=1/x, x\in(-\infty, 0) $, найти
	первообразную $ F(x) $, график которой проходит через
	точку $ (-2, 2) $.
	
	\item Найти интеграл:
	\begin{multicols}{3}
		\begin{enumerate}
			\item $ \displaystyle\int(x-2e^x)dx $.
			\item $ \displaystyle\int
			\dfrac{(\sqrt{x}-2\sqrt[3]{x})^2}{x}dx. $
			\item $ \displaystyle\int\dfrac{dx}{x^4+4x^2}. $
			\item $ \displaystyle\int
			\dfrac{\sqrt{x^2-3}-3\sqrt{x^2+3}}{\sqrt{x^4-9}}dx. $
			\item $ \displaystyle\int\cos^2\frac{x}{2}dx. $
			\item $ \displaystyle\int\tg^2xdx. $
			\item $ \displaystyle\int3^x\cdot5^{2x}dx. $
			\item $ \displaystyle\int(3x-5)^{10}dx. $
			\item $ \displaystyle\int x^2\sqrt[5]{5x^3+1}dx. $
			\item $ \displaystyle\int\tg xdx. $
			\item $ \displaystyle\int\dfrac{dx}{2+\cos^2x},
			~~\abs{x}<\frac{\pi}{2}. $
			\item $ \displaystyle\int 
			\dfrac{x^7dx}{\sqrt{1-x^{16}}}. $
			\item $ \displaystyle\int\ln xdx. $
			\item $ \displaystyle\int x\sin xdx. $
			\item $ \displaystyle\int x^2e^xdx. $
		\end{enumerate}
	\end{multicols}
	\item Частица массой $ m $, движущиеся горизонтально, с начальной
	скоростью $ v_0 $ попадает на поверхность, где на нее действует
	сила вязкого трения $ F=\alpha v $.
	Найти зависимость пройденного частицей пути от времени.
	\item Тело начинает падать с высоты $ H $
	под действием силы тяжести. В процессе падения оно
	испытывает сопротивление, пропорциональное скорости.
	Определить время падения.
\end{enumerate}

\end{document}




























