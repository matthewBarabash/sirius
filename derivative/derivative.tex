\documentclass[a4paper,12pt]{article} % добавить leqno в [] для нумерации слева

%%% Работа с русским языком
\usepackage{cmap}					% поиск в PDF
\usepackage{mathtext} 				% русские буквы в фомулах
\usepackage[T2A]{fontenc}		  	% кодировка
\usepackage[utf8]{inputenc}			% кодировка исходного текста
\usepackage[english,russian]{babel}	% локализация и переносы
\usepackage[left=2cm,right=2cm,top=2cm,bottom=2cm,footskip=1cm,includefoot
] {geometry}
%%% Дополнительная работа с математикой
\usepackage{amsmath,amsfonts,amssymb,amsthm,mathtools} % AMS
\usepackage{icomma} % "Умная" запятая: $0,2$ --- число, $0, 2$ --- перечисление

%% Номера формул
\mathtoolsset{showonlyrefs=true} % Показывать номера только у тех формул, на которые есть \eqref{} в тексте.

%% Перенос знаков в формулах (по Львовскому)
\newcommand*{\hm}[1]{#1\nobreak\discretionary{}
	{\hbox{$\mathsurround=0pt #1$}}{}}

%%% Работа с картинками
\usepackage{graphicx}  % Для вставки рисунков
\graphicspath{{images/}{images2/}}  % папки с картинками
\setlength\fboxsep{3pt} % Отступ рамки \fbox{} от рисунка
\setlength\fboxrule{1pt} % Толщина линий рамки \fbox{}
\usepackage{wrapfig} % Обтекание рисунков и таблиц текстом

%%% Работа с таблицами
\usepackage{array,tabularx,tabulary,booktabs} % Дополнительная работа с таблицами

\usepackage{longtable}  % Длинные таблицы
\usepackage{multirow} % Слияние строк в таблице
\newcommand{\RomanNumeralCaps}[1]
{\MakeUppercase{\romannumeral #1}}
\newcommand{\tb}[1]{\textbf{#1}}
\newcommand{\abs}[1]{\left|#1\right|}
\newcommand{\ol}[1]{\ensuremath{\overline{#1}}}

\usepackage{pgfplots, pgfplotstable}
\pgfplotsset{compat=1.9}
\usepackage{circuitikz}
\usepackage{ulem}
\usepackage{cancel}
\usepackage{tikz}
\usetikzlibrary{calc}
\usepackage{tkz-euclide}
\usepackage{multicol}
\usepackage{enumitem}
\begin{document} % конец преамбулы, начало документа

\begin{figure}[h!]
	\hfill
	\includegraphics[height=.09\textheight]{sirius}
\end{figure}
\vspace{5ex}
\begin{center}
	\Large\textbf{Производная.\\
		Формулы и правила вычисления производных} 
\end{center}

\paragraph*{Определение.}\textit{Производной} функции
$ f $ в точке $ x_0 $ называется 
\[
	\lim\limits_
	{x\rightarrow x_0}\dfrac{f(x)-f(x_0)}{x-x_0}=
	\lim\limits_{\Delta x\rightarrow0}
	\dfrac{f(x_0+\Delta x)-f(x_0)}{\Delta x}\in\mathbb{\overline{R}}
\]
и обозначается $ f'(x_0) $.
\paragraph{Теорема 1.}Если функции $ f $ и $ g $ 
дифференцируемы в точке $ x_0 $, то
\begin{enumerate}
	\item $ (f+g)'(x_0)=f'(x_0)+g'(x_0); $
	\item $ (f\cdot g)'(x_0)=f'(x_0)g(x_0)+g'(x_0)f(x_0). $
	\item Если дополнительно $ g(x_0)\ne0, $ то
	\[
		\left(\dfrac{f}{g}\right)'(x_0)=
		\dfrac{f'(x_0)g(x_0)-f(x_0)g'(x_0)}{g^2(x_0)}
	\]
\end{enumerate}
\paragraph{Теорема 2.}(Производная сложной функции).
Если функция $ y=f(x) $ имеет производную в точке $ x_0 $,
а функция $ z=g(y) $~--- в точке $ y_0=f(x_0) $, то сложная
функция $ z=\varphi(x)=g(f(x)) $ также имеет производную в
точке $ x_0 $, причем
\begin{equation}\label{comlderv}
	\varphi'(x_0)=g'(y_0)f'(x_0).
\end{equation}
Опуская аргумент и используя другое обозначение для 
производных, формулу \eqref{comlderv} можно переписать
в виде
\begin{equation}\label{varcomlderv}
	\dfrac{dz}{dx}=\dfrac{dz}{dy}\dfrac{dy}{dx}.
\end{equation}
\textbf{Пример.} Вычислить производную функции
$ z=\sqrt{1+x^2} $.\\
$ \blacktriangle $ Данная функция является композицией 
функций $ y=1+x^2 $ и $ z=\sqrt{y} $, причем 
\[
	\dfrac{dy}{dx}=2x~~~и~~~\dfrac{dz}{dy}=
	\dfrac{1}{2\sqrt{y}}.
\]
По формуле \eqref{varcomlderv} получаем 
\[
	\dfrac{dz}{dx}=\dfrac{1}{2\sqrt{y}}2x=
	\dfrac{x}{\sqrt{1+x^2}}.~\blacktriangle
\]
\paragraph{Теорема 3.} (Производная обратной функции).
Пусть $ \exists y'(x_0)\in\mathbb{R},~ y'(x_0)\ne0 $.
Тогда обратная функция $ x(y) $ дифференцируема в точке
$ y_0=u(x_0) $, причем
\begin{equation}\label{invfuncder}
	x'(y_0)=\dfrac{1}{y'(x_0)}.
\end{equation}
\textbf{Пример.} Найти производную функции $ y=\arcsin x $.\\
$ \blacktriangle $ Пользуясь формулой \eqref{invfuncder} и
определением арксинуса получаем
\[
	(\arcsin x)'=\dfrac{1}{(\sin y)'}=\dfrac{1}{\cos y}=
	\dfrac{1}{\sqrt{1-\sin^2y}}=\dfrac{1}{\sqrt{1-x^2}}. ~
	\blacktriangle
\]
\paragraph{Теорема 4.} (Производные элементарных функций).
\begin{multicols}{2}
	\begin{enumerate}\bfseries
		\item $ C'=0~~~(C=const); $
		\item $ (a^x)'=a^x\ln a, a>0,~ x\in\mathbb{R}; $
		\item $ (\log_a x)'=
		\dfrac{1}{x\ln a},~ a>0,~ a\ne1,~ x>0; $
		\item $ (x^\alpha)'=
		ax^{\alpha-1},~ \alpha\in\mathbb{R},~ x>0; $
		\item $ (\sin x)'=\cos x, ~~~ (\cos x)'=-\sin x; $
		\item $ (\tg x)'=
		\frac{1}{\cos^2x}, ~~~ (\ctg x)'=-\frac{1}{\sin^2x}; $
		\item $ (\arcsin x)'=\frac{1}{\sqrt{1-x^2}}, ~~
		(\arccos x)'=\frac{-1}{\sqrt{1-x^2}}; $
		\item $ (\arctg x)'=\frac{1}{1+x^2}, ~~~ 
		(\arcctg x)'=-\frac{1}{1+x^2} $
	\end{enumerate}
\end{multicols}
\section*{Задачи для работы в классе}

Вычислить производную функции $ y=f(x) $. Указать область
существования производной.
\begin{multicols}{3}
	\begin{enumerate}\bfseries
		\item $ y=ax^3+bx^2+cx+d. $
		\item $ y=\dfrac{\ln3}{x}+e^2.	 $
		\item 
		$ y=\dfrac{a}{x^2}+\dfrac{b}{x^3}+\dfrac{c}{x^4}. $
		\item $ y=(x+1)\tg x. $
		\item $ y=\dfrac{ax+b}{cx+d}. $
		\item $ y=x\arcsin x $
		\item $ y=(\sqrt{2})^x+(\sqrt{5})^{-x}. $
		\item $ y=(x^2-7x+8)e^x. $
		\item $ y=\log_x2^x  $
	\end{enumerate}
\end{multicols}
\begin{enumerate}[label=\textbf{\arabic*.}]
	\setcounter{enumi}{6}
	\item Снаряд вылетел с начальной скоростью $ v_0 $ под
	углом $ \alpha $ к горизонту. В какой момент скорость
	изменения высоты снаряда равна нулю.
	\item Количество электричества $ q $ (в кулонах),
	протекающее через поперечное сечение проводника,
	изменяется по закону $ q=3t^2+2t $. Найти силу тока
	в конце пятой секунды.
	\item Колесо вращается так, что угол поворота
	 пропорционален квадрату времени. Первый оборот бы сделан
	 за 8~с. Найти угловую скорость через 64~c после начала
	 движения.
	 \item Масса $ m(t) $ радиоактивного вещества изменяется
	 по закону $ m=m_02^{(t_0-t)/T} $, где $ t $~--- время,
	 $ m_0 $~--- масса в момент времени $ t_0 $, $ T $~---
	 период полураспада. Доказать, что скорость распада
	 радиоактивного вещества пропорциональна количеству
	 вещества. Найти коэффициент пропорциональности.
	 \item Определить при каком соотношении сопротивлений,
	 последовательно соединенных с источником питания, на них выделяется максимальная мощность.
	 \item Свойства идеального газа описывается уравнением Менделеева-Клапейрона $ pV=\nu RT $. Определить 
	 максимальную температуру в процессе 
	 $ (V_0, 3p_0)\rightarrow(3V_0, p_0) $.
\end{enumerate}

\end{document}




























